%%=====================================================================================
%%
%%       Filename:  arrow.tex
%%
%%    Description:  
%%
%%        Version:  1.0
%%        Created:  05/04/2017
%%       Revision:  none
%%
%%         Author:  Dilawar Singh (), dilawars@ncbs.res.in
%%   Organization:  NCBS Bangalore
%%      Copyright:  Copyright (c) 2017, Dilawar Singh
%%
%%          Notes:  
%%                
%%=====================================================================================

\RequirePackage{luatex85}
\documentclass[crop,tikz]{standalone}
\usepackage{pgfplots}
\usepackage{tikz}
\usetikzlibrary{arrows,arrows.meta}
\begin{document}

\pgfdeclarearrow{
    name = foo,
    parameters = { \the\pgfarrowlength },
    setup code = {
        % The different end values:
        \pgfarrowssettipend{.25\pgfarrowlength}
        \pgfarrowssetlineend{-.25\pgfarrowlength}
        \pgfarrowssetvisualbackend{-.5\pgfarrowlength}
        \pgfarrowssetbackend{-.75\pgfarrowlength}
        % The hull
        \pgfarrowshullpoint{.25\pgfarrowlength}{0pt}
        \pgfarrowshullpoint{-.75\pgfarrowlength}{.5\pgfarrowlength}
        \pgfarrowshullpoint{-.75\pgfarrowlength}{-.5\pgfarrowlength}
        % Saves: Only the length:
        \pgfarrowssavethe\pgfarrowlength
    },
    drawing code = {
        \pgfsetdash{}
        \pgftext[left]{$X$}
        \pgfusepathqfill
    },
    defaults = {length=4}
}

\begin{tikzpicture}[scale=1, every node/.style={} ]
    \node[thick] at (0,0) {+};
    \draw[gray!20] (-5,-5) grid (5,5);
    \draw[foo-foo] (0,0) -- (3cm,0);
    %\draw[-foo] (0,1) -- (10cm,1);
\end{tikzpicture}    

\end{document}

    
