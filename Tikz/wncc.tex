%%=====================================================================================
%%
%%       Filename:  wncc.tex
%%
%%    Description:  
%%
%%        Version:  1.0
%%        Created:  04/17/2016
%%       Revision:  none
%%
%%         Author:  Dilawar Singh (), dilawars@ncbs.res.in
%%   Organization:  NCBS Bangalore
%%      Copyright:  Copyright (c) 2016, Dilawar Singh
%%
%%          Notes:  
%%                
%%=====================================================================================

\documentclass[crop,tikz]{standalone}
\usetikzlibrary{positioning}
\begin{document}

% Here we go.
% Ruben is from tikzexample.

\def\arete{3}   
\def\epaisseur{5}   
\def\rayon{2}

\newcommand{\ruban}{(0,0)
  ++(0:0.57735*\arete-0.57735*\epaisseur+2*\rayon)
  ++(-30:\epaisseur-1.73205*\rayon)
  arc (60:0:\rayon)   -- ++(90:\epaisseur)
  arc (0:60:\rayon)   -- ++(150:\arete)
  arc (60:120:\rayon) -- ++(210:\epaisseur)
  arc (120:60:\rayon) -- cycle}
   
\begin{tikzpicture}[very thick,top color=white,bottom color=gray]
  \shadedraw \ruban;
  \shadedraw [rotate=120] \ruban;
  \shadedraw [rotate=-120] \ruban;
  \draw (-60:4) node[scale=5,rotate=30]{$< \ldots >$};
  \draw (180:4) node[scale=3,rotate=-90]{$(\lambda x \rightarrow x ^ 2) a$ };
  \clip (0,-6) rectangle (6,6);
  \shadedraw  \ruban;
  \draw[align=left] (60:4) node [black!80,xscale=-2,yscale=2,rotate=30]{
          \scriptsize 
          \textbf{do} \{ \\
              \ldots  \\
              \} \textbf{while} \{ \\
              \ldots 
          \}
      };
\end{tikzpicture}


\end{document}

