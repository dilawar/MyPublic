\documentclass[10pt,sans,serif,trans]{beamer}
\usepackage{beamerthemeshadow}
\usepackage{fontenc}
\usepackage{tkz-berge}
\usepackage{graphicx}
\title{Van-Ginneken's dynamic programming for Buffer Insertion Problem}
% \\ \& \\Graphs and Laplacian Systems}
\institute[EE, IIT Bombay]{Department of Electrical Engineering \\  Indian
Institute of Technology Bombay}
\author{
Aayush Rathi \\
Ankit Mehta \\
Dilawar Singh}
\date{\today}
\begin{document}
\begin{frame}
   \maketitle
\end{frame}

%%%%%%%%%%%%%%%%%%%%%%%%%%%%%%%%%
\begin{frame}{Abstact}
   \begin{itemize}
      \item An algorithm for choosing locations for buffer insertion such that Elmore delay (delay
is through RC Network) \footnote{For tree structured networks, find the delay through each segment
as the R (electrical resistance) times the downstream C (electrical capacitance). Sum the delays
from the root to the sink.}is minimum.
\item For given required arrival times at the sinks of the wiring tree, the departure time at the
source is scheduled as late as possible.
\item Basic Idea: Using DFS on the wiring tree, construct a set of time-capacitance pairs
corresponding to different choices.
\end{itemize}

\begin{columns}[t]
\begin{column}[T]{5cm}
\begin{figure}[h]
\centering
\includegraphics[height=3cm, width = 4cm]{./images/tree.jpeg} 
\caption{RC Tree Network with discrete elements.}
\end{figure}
\end{column}
\begin{column}[T]{5cm}
\begin{figure}[h]
\centering
\includegraphics[width=4cm, height=2cm]{./images/s_tree.jpeg}
\caption{A steiner tree. A source and two sink}
\end{figure}
\end{column}
\end{columns}

\end{frame}

%%%%%%%%%%%%%%%%%%%%%%%%
\begin{frame}{Need for buffer insertion}
 \begin{itemize}
 \item Buffer insertion and Interconnect Topology optimizations have an important role for Timing
optimizations of VLSI circuits.
    \item With gate delay decreasing with each new technology node, delay in the charging and
discharging of wire capacitance has started to dominate.
\item Buffering decreases the growth of delay with the length of the wire from quadratic to linear
 \end{itemize}
\end{frame}
%%%%%%%%%%%%%%%%%%%%%%%%%%%%%%%%%%5
\begin{frame}{Statement of the Problem}
%    \begin{definition}
\begin{block}{Problem}
      On the given wiring tree, there are certain legal positions for buffer insertion. Defer the
required departure time at the $source$, $T_{source}$, to as late as possible, \\
\begin{equation}
T_{source} = min_i(T_i-D_i)    
\end{equation}
	where $T_i$ = required arrival time of signal at each sink and $D_i$ = delay between the source
of the signal and the sink $i$.
\end{block}
\begin{figure}[h]
\centering
\includegraphics[height=3cm, width = 4cm]{./images/tree.jpeg} 
\caption{RC Tree Network with discrete elements.}
\end{figure}
\end{frame}
%%%%%%%%%%%%%%%%%%%%%%%%%%%%%%%%%%%5
\begin{frame}{Delay Calculation}
\begin{columns}[t] % contents are top vertically aligned
\begin{column}[T]{7cm} % each column can also be its own environment
\begin{itemize}
  \item \color{red}Exact delay is difficult since it requires calculation will require solving
system of differential equations. \color{black}
    \item However, it is possible to derive easily computed estimate of delay.
     \item Capacitance of surrounding wires is modeled as extra contribution to ground capacitance.
   
 \end{itemize}
\end{column}

\begin{column}[T]{5 cm}
\begin{figure}[h]
\centering
\includegraphics[height=3cm, width = 4cm]{./images/tree.jpeg}
\caption{RC Tree Network with discrete elements.}
\end{figure}
\end{column}
\end{columns}
\begin{itemize}
   \item Root of the tree is source of the signal and leaves are the sinks.
 \item Thc wiring tree are modeled by a tree of distributed RC sections.
\item For tree structured networks, find the delay through each segment
as the R (electrical resistance) times the downstream C (electrical capacitance). Sum the delays
from the root to the sink.
 
\end{itemize}

\end{frame}
%%%%%%%%%%%%%%%%%%%%%%%%%%%%%%%%%%%%5
%%%%%%%%%%%%%%%%%%%%%%%%%%%%%%%%%%%
\begin{frame}{Van Ginnekan Algorithm}
\begin{itemize}
   \item The input is a wiring tree with estimated parasitics and a set of possible
buffer locations, fanout capacitive loadings and required arrival times. 
\item The wire parasitics in turn are dependent on the wire-lengths which are
assumed to be pre-specified. 
\item The problem is to decide buffers locations at the pre-specified positions such that the
required time at the root is maximized. 
\item \textbf{Traverses the wiring tree topologically from primary outputs to primary inputs. At
each possible buffer location, it evaluates the possibility of adding a buffer and its
effect on the required time and capacitive loading at that node.}
\item This is followed by local pruning of the generated solutions. The pruning
criteria removes those solutions from the set of potential solutions
which have another solution with better values of both required time.
 \footnote{Buffer Placement in Distributed RC-tree Networks for Minimal Elmore Delay, Lukas P.P.P.
van Ginncken}  
 \end{itemize}
\end{frame}
%%%%%%%%%%%%%%%%%%%%%%%%%%%%%%%%%%%%
\begin{frame}{Van Ginnekan Algorithm}
\begin{figure}[h]
   \centering
   \includegraphics[width=\textwidth]{./images/diag1.jpg}
\end{figure}
\end{frame}
\begin{frame}{Van Ginnekan Algorithm}
\begin{figure}[h]
   \centering
   \includegraphics[width=\textwidth]{./images/diag2.jpg}
\end{figure}
Pick Solution with largest slack. Follow arrow to get the solutions.
\end{frame}
%%%%%%%%%%%%%5
\begin{frame}{Challenges}
   \begin{itemize}
      \item It was assumed that wire-lengths of individual segments in the wiring tree are known
a-priori through some estimation engine. 
\item Estimating wirelengths especially in a traditional top down design flow is extremely hard and
error prone. This makes estimation of delay and capacitive loading of
the individual wire segments extremely difficult. 
\item Even if the wire-length estimates are accurate, the fabrication variability/uncertainty
makes accurate estimation of wire parasitics an intractable problem. The traditional
deterministic approach possesses serious disadvantages.
\item Van Ginneken’s algorithm is optimal only under the condition that at most one
buffer may be placed on each wire. This constraint will severely restrict the solution space when
multiple buffers are required to effectively drive wires with large lumped RCs.\footnote{Wire
Segmenting for Improved Buffer Insertion, Charles Alpert and Anirudh Devgan, DAC, 97}
\item People have shown better algorithms
using probabilistic models.\footnote{A Probabilistic Approach to Buffer Insertion, Vishal Khandelwal
et. all, ICCAD 03}
\end{itemize}
\end{frame}

\end{document}